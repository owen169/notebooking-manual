\documentclass[letterpaper, 12pt]{article}

% Set Margins
\usepackage[margin=1in, includefoot]{geometry}

% Color Text
\usepackage{xcolor}

% Set Headers
\usepackage{fancyhdr}
\pagestyle{fancy}
\fancyhead{}
\fancyfoot{}
\fancyfoot{}
\renewcommand{\headrulewidth}{0px}

\begin{document}
\begin{center}
\huge\textsc{Vex Notebooking Manual}
\line(1,0){400}

\vspace{0.1in}

\end{center}

Hard work is the foundation that you build your knowledge and skills upon. Experience comes with effort, time, and lots of practice. Even a completely new team can be successful with the right mindset. This guide exists to help anyone with any amount of experience create a successful Engineering Notebook.

The Engineering Notebook should log your build process and thoughts as a team. This will help you organize ideas and write down successes/failures to learn from in the future. Most judged awards require teams to submit a notebook. The Design and Excellence awards can qualify to the world championships at certain large tournaments like state championships, national events, and signature events, so most successful teams take their Engineering Notebook seriously. A successful engineering notebook should include the following sections.

\section*{Cover}

The cover doesn’t need much information, but it should be obvious which team the notebook belongs to.

\section*{Table of Contents}

Make sure to include titles, dates, and optionally label sections. This makes it easier to navigate your notebook.

\section*{Biography}

The biography is where team members explain their past experiences in robotics and what they plan to do on the team. Members may also include what other activities they do, such as sports, student council, or volunteer work. This helps the judges get to know the members on the team and how you all work together. Try to keep biographies short and simple, but detailed enough so the reader can understand your team dynamic.

\section*{Team Goals}

At the beginning of the season, teams should set their goals so they understand the focus of their robot. Some teams want to build a unique design while others want to perform as well as they can in tournaments. It is important that the team agrees, since major conflict can be avoided in the future if they agree on an end goal.

\section*{Game Analysis}

Before designing or building a robot, teams must understand how to score points effectively. It is easy to skip game analysis and immediately think of robot ideas, but teams should understand how the game works before designing a robot. Teams should discuss point values and strategies here.

\section*{Design Ideas}

After analyzing the game to understand how to score points efficiently, teams must think of robot designs to accomplish the task. Individual team members should present their ideas and no design should be left out. After listing the designs, team members should compare and choose the most efficient design to build for their first robot. Don’t be afraid to spend a lot of time choosing a design, since it prevents switching designs in the future, effectively losing the development of the previous robot. Improve upon the robot you already have, since going through multiple iterations raises your chances of success.

\section*{Build Logs}

Build logs take up the largest chunk of the notebook and are often the most difficult to stay consistent with. Because of this, teams should decide how much time they want to invest in the notebook and plan the frequency of entries accordingly. Weekly build logs that are 1-2 pages are a good starting point, but teams can write as often as they want, as long as the entries are high quality and consistent. Adding pictures, graphs, diagrams, CAD drawings, and hand-drawn drawings alongside your entries is a good way to show progress.

\section*{Recaps}

After each tournament, teams should go over their successes and failures to improve in the future. Write down what you learned from other teams and organize the data from scouting sheets to plan for the next competition. Optionally, teams can also do monthly entries to go over the progress they made in that month.

\vspace{1.2in}

\begin{center}
\textcolor{darkgray}{This guide was written by members from Team 169 and Team 7700}

\vspace{0.1in}

\textcolor{darkgray}{March 2021}
\end{center}

\end{document}